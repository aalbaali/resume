%%%%%%%%%%%%%%%%%
% This is an sample CV template created using altacv.cls
% (v1.1.3, 30 April 2017) written by LianTze Lim (liantze@gmail.com). Now compiles with pdfLaTeX, XeLaTeX and LuaLaTeX.
% 
%% It may be distributed and/or modified under the
%% conditions of the LaTeX Project Public License, either version 1.3
%% of this license or (at your option) any later version.
%% The latest version of this license is in
%%    http://www.latex-project.org/lppl.txt
%% and version 1.3 or later is part of all distributions of LaTeX
%% version 2003/12/01 or later.
%%%%%%%%%%%%%%%%

%% If you need to pass whatever options to xcolor
\PassOptionsToPackage{dvipsnames}{xcolor}

%% If you are using \orcid or academicons
%% icons, make sure you have the academicons 
%% option here, and compile with XeLaTeX
%% or LuaLaTeX.
% \documentclass[10pt,a4paper,academicons]{altacv}

%% Use the "normalphoto" option if you want a normal photo instead of cropped to a circle
% \documentclass[10pt,a4paper,normalphoto]{altacv}

\documentclass[10pt,letter]{altacv}
%% AltaCV uses the fontawesome and academicon fonts
%% and packages. 
%% See texdoc.net/pkg/fontawecome and http://texdoc.net/pkg/academicons for full list of symbols.
%% 
%% Compile with LuaLaTeX for best results. If you
%% want to use XeLaTeX, you may need to install
%% Academicons.ttf in your operating system's font 
%% folder.


% Change the page layout if you need to
\geometry{left=1cm,right=9cm,marginparwidth=6.8cm,marginparsep=1.2cm,top=1.25cm,bottom=1.25cm,footskip=2\baselineskip}
\definecolor{MyBackground}{RGB}{255,241,167}
% Change the font if you want to.

% If using pdflatex:
\usepackage[T1]{fontenc}
\usepackage[utf8]{inputenc}
\usepackage[default]{lato}

% If using xelatex or lualatex:
% \setmainfont{Lato}

% Change the colours if you want to
\definecolor{Mulberry}{HTML}{72243D}
\definecolor{SlateGrey}{HTML}{2E2E2E}
\definecolor{LightGrey}{HTML}{666666}

\definecolor{indigo(dye)}{rgb}{0.0, 0.25, 0.42}
\definecolor{lapislazuli}{rgb}{0.15, 0.38, 0.61}
\definecolor{custombody}{HTML}{4A4843}

\colorlet{heading}{indigo(dye)}
\colorlet{accent}{lapislazuli}
\colorlet{emphasis}{black}
\colorlet{body}{black}

% Change the bullets for itemize and rating marker
% for \cvskill if you want to
\renewcommand{\itemmarker}{{\small\textbullet}}
\renewcommand{\ratingmarker}{\faCircle}

%% sample.bib contains your publications
\addbibresource{sample.bib}
\DeclareNameAlias{sortname}{first-last}

\usepackage[colorlinks]{hyperref}
\definecolor{linkcolour}{rgb}{0,0.2,0.55}
\hypersetup{colorlinks,breaklinks,urlcolor=linkcolour, linkcolor=black}

\begin{document}

\name{Amro Al-Baali}
\tagline{M.\,Eng Mechanical Engineering}

\personalinfo{%
  % Not all of these are required!
  % You can add your own with \printinfo{symbol}{detail}
  \email{amro.albaali@gmail.com}
  \phone{+1 (514) 296-1957}
  \location{Canada}\\
  \homepage{\href{http://aalbaali.github.io}{aalbaali.github.io}}
  \linkedin{\href{http://www.linkedin.com/in/amro-al-baali}{www.linkedin.com/in/amro-al-baali}}
  \github{\href{http://github.com/aalbaali}{github.com/aalbaali}}
  %% You MUST add the academicons option to \documentclass, then compile with LuaLaTeX or XeLaTeX, if you want to use \orcid or other academicons commands.
%   \orcid{orcid.org/0000-0000-0000-0000}
}

%% Make the header extend all the way to the right, if you want. 
\begin{fullwidth}
\makecvheader
\end{fullwidth}

%% Education
\cvsection[page1sidebar]{Education}

\cvevent{M.\,Eng\ Mechanical}{McGill University}{Graduated May 2021}{}
\begin{itemize}
\item CGPA: 3.77/4.00.
\item Thesis title: \emph{Augmenting Sensor Measurements with INS Estimates in a Graph Based SLAM Problem}.
\item Supervisor: \href{https://www.mcgill.ca/mecheng/james-forbes}{\textsc{Prof.\,J.\,R.\,Forbes}}.
\end{itemize}

\divider

\cvevent{B.\,Eng\ Honours Mechanical, \\Minor in Computer Science}{McGill University}{Graduated May 2019}{}
\begin{itemize}
\item CGPA: 3.83/4.00. Dean's Honour List 2015, 2018. 
%\item Thesis on control of non-minimum phase systems.
\item Supervisor: \href{https://www.mcgill.ca/mecheng/james-forbes}{\textsc{Prof.\,J.\,R.\,Forbes}}.
\end{itemize}

%% Provide the file name containing the sidebar contents as an optional parameter to \cvsection.
%% You can always just use \marginpar{...} if you do
%% not need to align the top of the contents to any
%% \cvsection title in the "main" bar.
\cvsection{Experience}

\cvevent{Software Developer - Localization and Mapping}{\href{https://avidbots.com/}{Avidbots}}{09/2021 -- Present}{Kitchener, Canada}
Develop and maintain the calibration, localization, and mapping algorithms for a robot equipped with a 2D LIDAR and a camera such that it is well localized within a pre-defined map.
The primary tools used in this job are \textbf{ROS}, \textbf{C++}, \textbf{Python}, and nonlinear least squares (mainly using \href{https://github.com/ceres-solver/ceres-solver}{Ceres}).

\divider

\cvevent{Graduate Student - SLAM}{\href{https://www.decar.ca}{\textsc{DECAR group}}}{05/2019 -- 08/2021}{Montreal, Canada}
Developed a SLAM algorithm for an AUV equipped with a third-party INS treated as a black box and a high-precision laser scanner.
The primary tools used in the project are:
Lie groups, state estimation, optimization (convex, on-manifold), \textbf{MATLAB}, and \textbf{C++}.

\divider

\cvevent{Mechanical Engineering Intern}{\href{https://my01.io/}{MY01}}{05/2018 -- 04/2019}{Montreal, Canada}
Designed and executed mechanical tests on the MY01 device.
This included programming the testing platform using \textbf{Python} and customizing Autodesk Vault using \textbf{C\#}.

\divider

\cvevent{Undergraduate Researcher Assistant}{\href{https://www.decar.ca}{\textsc{DECAR group}}}{09/2017 -- 05/2018}{Montreal, Canada}
Developed a systematic method of controlling a non-minimum phase system with minimal
effect on the performance of the system.
\textbf{MATLAB} was used in this project (control toolbox, LMIs, optimization).

\divider

\cvevent{Teaching Assistant}{McGill University}{09/2017 -- 04/2021}{Montreal, Canada}
\begin{itemize}[noitemsep]
  \item \href{https://www.mcgill.ca/study/2022-2023/courses/mech-513}{MECH 513 (Control Systems)} (Winter 2021)
  \item \href{https://www.mcgill.ca/study/2022-2023/courses/mech-309}{MECH 309 (Numerical Methods)} (Fall 2019)
  \item \href{https://www.mcgill.ca/study/2022-2023/courses/mech-412}{MECH 412 (System Dynamics and Control)} (Fall 2017)
\end{itemize}


\end{document}
